% (c) Andrei Ruslantsev 2025
% Based on Reference Card for C by prof. Joseph H. Silverman and https://en.cppreference.com

% TODO: literals 0xDDD, 0DDD, l, i, ...

\documentclass{article}
\usepackage[fontsize=9pt]{scrextend}
\usepackage[a4paper,landscape,left=1cm,right=1cm,top=1cm,bottom=1cm]{geometry}
\usepackage[utf8]{inputenc}
\usepackage{multicol}
\usepackage{xspace}
\usepackage{supertabular}

\setlength{\parindent}{0pt}
\emergencystretch=10pt

% \newcommand{<name>}[<args>]{ <code> }
\newcommand{\mktitle}[1]{{\Large \begin{center}\textbf{#1}\end{center}}\par \vspace{2ex}}
\newcommand{\mksection}[1]{{\vspace{2ex}\large \textbf{#1}}\par \vspace{2ex}}
\newcommand{\newstd}{\ensuremath{^{(C99)}}\xspace}

\newcommand{\librarysection}[5]{{\vspace{2ex}\large #1\quad\textbf{\texttt{#2}}}\par\begin{supertabular}{p{#3\linewidth}p{#4\linewidth}}#5\end{supertabular}}
\newcommand{\funcdescription}[2]{\texttt{#1} & #2 \\}
\newcommand{\smallheader}[1]{\multicolumn{2}{c}{#1} \\}

% Wrap table over columns
% https://tex.stackexchange.com/questions/105717/trick-supertabular-into-multicols-in-new-command
\newcount\n
\n=0
\def\tablebody{}
\makeatletter
\loop\ifnum\n<100
        \advance\n by1
        \protected@edef\tablebody{\tablebody
                \textbf{\number\n.}& shortText
                \tabularnewline
        }
\repeat

\makeatletter
\let\mcnewpage=\newpage
\newcommand{\TrickSupertabularIntoMulticols}{%
  \renewcommand\newpage{%
    \if@firstcolumn
      \hrule width\linewidth height0pt
      \columnbreak
    \else
      \mcnewpage
    \fi
  }%
}
\makeatother


\begin{document}
\begin{multicols*}{3}
\TrickSupertabularIntoMulticols
\mktitle{C Reference Card (ANSI, C99)}

\mksection{Keywords}
\texttt{auto, break, case, char, const, continue, default, do, double, else,
enum, extern, float, for, goto, if, inline\newstd, int, long, register,
restrict\newstd, return, short, signed, sizeof, static, struct, switch,
typedef, union, unsigned, void, volatile, while}


\mksection{Standard libraries}
\texttt{<assert.h> <complex.h>\newstd <ctype.h> <errno.h> <fenv.h>\newstd <float.h>
<inttypes.h>\newstd <iso646.h> <limits.h> <locale.h> <math.h> <setjmp.h>
<signal.h> <stdarg.h> <stdbool.h>\newstd <stddef.h> <stdint.h>\newstd <stdio.h>
<stdlib.h> <string.h> <tgmath.h>\newstd <time.h> <wchar.h>\newstd <wctype.h>\newstd}

\librarysection{Asserts}{assert.h}{0.2}{0.7}{
\funcdescription{assert (cond)}{abort the program if \textit{cond} is not true; skipped if defined \texttt{NDEBUG}}
}

\librarysection{Complex numbers}{complex.h\newstd}{0.2}{0.7}{
\smallheader{\underline{Types}}
\funcdescription{imaginary}{imaginary type, use as \texttt{float imaginary}, \texttt{double imaginary}, \texttt{long double imaginary}}
\funcdescription{complex}{complex type, use with float types as \texttt{imaginary}}
\smallheader{\underline{Constants}}
\funcdescription{I}{the complex or imaginary unit constant i }
\smallheader{\underline{Functions}}
\funcdescription{creal, crealf, creall}{real part}
\funcdescription{cimag$^{(*)}$}{imaginary part}
\smallheader{$^{(*)}$ also presented for \texttt{float} and \texttt{long double}}
\funcdescription{cabs$^{(*)}$}{magnitude}
\funcdescription{carg$^{(*)}$}{phase angle}
\funcdescription{conj$^{(*)}$}{complex conjugate}
\funcdescription{cproj$^{(*)}$}{projection of Riemann sphere}
\funcdescription{cexp$^{(*)}$}{complex exponential}
\funcdescription{clog$^{(*)}$}{complex natural logarithm}
\funcdescription{cpow$^{(*)}$}{complex power}
\funcdescription{csqrt$^{(*)}$}{complex square root}
\funcdescription{csin$^{(*)}$}{complex sine}
\funcdescription{ccos$^{(*)}$}{complex cosine}
\funcdescription{ctan$^{(*)}$}{complex tangent}
\funcdescription{casin$^{(*)}$}{complex arc sine}
\funcdescription{cacos$^{(*)}$}{complex arc cosine}
\funcdescription{catan$^{(*)}$}{complex arc tangent}
\funcdescription{csinh$^{(*)}$}{complex hyperbolic sine}
\funcdescription{ccosh$^{(*)}$}{complex hyperbolic cosine}
\funcdescription{ctanh$^{(*)}$}{complex hyperbolic tangent}
\funcdescription{casinh$^{(*)}$}{complex arc hyperbolic sine}
\funcdescription{cacosh$^{(*)}$}{complex arc hyperbolic cosine}
\funcdescription{catanh$^{(*)}$}{complex arc hyperbolic tangent}
}

\librarysection{Character class tests}{ctype.h}{0.25}{0.65}{
\funcdescription{isalnum}{alphanumeric?}
\funcdescription{isalpha}{alphabetic?}
\funcdescription{islower}{lower case letter?}
\funcdescription{isupper}{upper case letter?}
\funcdescription{isdigit}{decimal digit?}
\funcdescription{isxdigit}{hexadecimal digit?}
\funcdescription{iscntrl}{control character?}
\funcdescription{isgraph}{printing character (not incl space)?}
\funcdescription{isspace}{space, formfeed, newline, cr, tab, vtab?}
\funcdescription{isblank\newstd}{blanc character?}
\funcdescription{isprint}{printing character (incl space)?}
\funcdescription{ispunct}{printing char except space, letter, digit?}
\funcdescription{tolower}{convert to lower case}
\funcdescription{toupper}{convert to upper case}
}

\librarysection{Error handling}{errno.h}{0.3}{0.6}{
\smallheader{\underline{Macros}}
\funcdescription{errno}{error number}
\funcdescription{E2BIG, EACCES, \dots, EXDEV}{standard POSIX-compatible error conditions}
}

\librarysection{Floating point environment}{fenv.h\newstd}{0.3}{0.6}{
\smallheader{\underline{Types}}
\funcdescription{fenv\_t}{entire floating-point environment}
\funcdescription{fexcept\_t}{all floating-point status flags collectively}
\smallheader{\underline{Functions}}
\funcdescription{feclearexcept}{clear the specified FP status flags}
\funcdescription{fetestexcept}{determine which of the specified FP status flags are set}
\funcdescription{feraiseexcept}{raise the specified FP exceptions}
\funcdescription{fegetexceptflag, fesetexceptflag}{copy the state of the specified FP status flags from or to the FP env}
\funcdescription{fegetround, fesetround}{get or set rounding direction}
\funcdescription{fegetenv, fesetenv}{save or restore the current FP env}
\funcdescription{feholdexcept}{save the env, clear all status flags and ignore all future errors}
\funcdescription{feupdateenv}{restore the FP env and raise the previously raised exceptions}
\smallheader{\underline{Macros}}
\funcdescription{FE\_ALL\_EXCEPT, FE\_DIVBYZERO, FE\_INEXACT, FE\_INVALID, FE\_OVERFLOW, FE\_UNDERFLOW}{FP exceptions}
\funcdescription{FE\_DOWNWARD, FE\_TONEAREST, FE\_TOWARDZERO, FE\_UPWARD}{rounding direction}
\funcdescription{FE\_DFL\_ENV}{default FP env}
}

\librarysection{Float type limits}{float.h}{0.3}{0.6}{
\funcdescription{FLT\_RADIX}{the radix (integer base) used by the representation of all floating-point types}
\funcdescription{DECIMAL\_DIG\newstd}{decimal precision required to (de)serialize long double}
\funcdescription{FLT\_MIN, DBL\_MIN, LDBL\_MIN}{minimum normalized positive \texttt{float}, \texttt{double}, \texttt{long double} value}
\funcdescription{FLT\_MAX$^{(*)}$}{maximum finit value of \texttt{float}, \texttt{double}, \texttt{long double}}
\smallheader{$^{(*)}$ also presented for \texttt{double} and \texttt{long double}}
\funcdescription{FLT\_EPSILON$^{(*)}$}{smallest $x$ so $1.0{\rm f}+x\ne1.0{\rm f}$}
\funcdescription{FLT\_DIG$^{(*)}$}{number of decimal digits that are guaranteed to be preserved in text - float - text roundtrip}
\funcdescription{FLT\_MANT\_DIG$^{(*)}$}{number of base-\texttt{FLT\_RADIX} digits that are in the floating-point mantissa}
\funcdescription{FLT\_MIN\_EXP$^{(*)}$}{minimum exponent}
\funcdescription{FLT\_MIN\_10\_EXP$^{(*)}$}{minimum exponent}
\funcdescription{FLT\_MAX\_EXP$^{(*)}$}{maximum exponent}
\funcdescription{FLT\_MAX\_10\_EXP$^{(*)}$}{maximum exponent}
\funcdescription{FLT\_ROUNDS}{floating point rounding mode}
\funcdescription{FLT\_EVAL\_METHOD\newstd}{specifies in what precision all arithmetic operations are done}
}

\librarysection{Integer Types}{inttypes.h\newstd}{0.2}{0.7}{
\smallheader{\underline{Types}}
\funcdescription{imaxdiv\_t}{struct, contains quot and rem (result of division)}
\smallheader{\underline{Functions}}
\funcdescription{imaxabs}{absolute value}
\funcdescription{imaxdiv}{division, returns \texttt{imaxdiv\_t}}
\funcdescription{strtoimax}{string to integer}
\funcdescription{strtoumax}{string to unsigned integer}
\funcdescription{wcstoimax}{wide characters to integer}
\funcdescription{wcstoumax}{wide characters to unsogned integer}
}

\librarysection{ISO 646}{iso646.h\newstd}{0.3}{0.6}{
\smallheader{\underline{Macros}}
\funcdescription{and}{\texttt{\&\&}}
\funcdescription{and\_eq}{\texttt{\&=}}
\funcdescription{bitand}{\texttt{\&}}
\funcdescription{bitor}{\texttt{|}}
\funcdescription{compl}{\texttt{~}}
\funcdescription{not}{\texttt{!}}
\funcdescription{not\_eq}{\texttt{!=}}
\funcdescription{or}{\texttt{||}}
\funcdescription{or\_eq}{\texttt{|=}}
\funcdescription{xor}{\texttt{\^}}
\funcdescription{xor\_eq}{\texttt{\^}\texttt{=}}  % TODO: FIX
}

\librarysection{Integer Type Limits}{limits.h}{0.4}{0.5}{
\funcdescription{CHAR\_BIT}{bits in \texttt{char}}
\funcdescription{MB\_LEN\_MAX}{maximum number of bytes in a multibyte character}
\funcdescription{CHAR\_MIN}{min value of \texttt{char}}
\funcdescription{CHAR\_MAX}{max value of \texttt{char}}
\funcdescription{SCHAR\_MIN, SHRT\_MIN, INT\_MIN, LONG\_MIN, LLONG\_MIN\newstd}{minimum value for signed types}
\funcdescription{SCHAR\_MAX, SHRT\_MAX, INT\_MAX, LONG\_MAX, LLONG\_MAX\newstd}{maximum value for signed types}
\funcdescription{UCHAR\_MAX, USHRT\_MAX, UINT\_MAX, ULONG\_MAX, ULLONG\_MAX\newstd}{maximum value for unsigned types}
}

\librarysection{Localization}{locale.h}{0.35}{0.55}{
\smallheader{\underline{Types}}
\funcdescription{lconv}{formatting details, returned by localeconv}
\smallheader{\underline{Constants}}
\funcdescription{NULL}{implementation-defined null pointer constant}
\funcdescription{LC\_ALL, LC\_COLLATE, LC\_CTYPE, LC\_MONETARY, LC\_NUMERIC, LC\_TIME}{locale categories for setlocale}
\smallheader{\underline{Functions}}
\funcdescription{setlocale}{gets and sets the current C locale}
\funcdescription{localeconv}{queries numeric and monetary formatting details of the current locale}
}

\librarysection{Mathematical Functions}{math.h}{0.35}{0.55}{
\smallheader{\underline{Types}}
\funcdescription{float\_t\newstd}{most efficient floating-point type at least as wide as float}
\funcdescription{double\_t\newstd}{most efficient floating-point type at least as wide as double}
\smallheader{\underline{Constants}}
\funcdescription{HUGE\_VALF\newstd, HUGE\_VAL, HUGE\_VALL\newstd}{indicates value too big to be representable (infinity)}
\funcdescription{INFINITY\newstd}{evaluates to positive infinity or the value guaranteed to overflow a float}
\funcdescription{NAN\newstd}{evaluates to a quiet NaN of type float}
\funcdescription{FP\_FAST\_FMA\newstd, FFP\_FAST\_FMA\newstd, FP\_FAST\_FMAL\newstd}{indicates that the fma function generally executes about as fast as, or faster than, a multiply and an add of double operands}
\funcdescription{FP\_ILOGB0\newstd, FP\_ILOGBNAN\newstd}{evaluates to \texttt{ilogb(x)} if x is zero or NaN, respectively}
\funcdescription{math\_errhandling\newstd, MATH\_ERRNO\newstd, MATH\_ERREXCEPT\newstd}{defines the error handling mechanism used by the common mathematical functions}
\funcdescription{FP\_NORMAL\newstd, FP\_SUBNORMAL\newstd, FP\_ZERO\newstd, FP\_INFINITE\newstd, FP\_NAN\newstd}{indicates a floating-point category}
\smallheader{\underline{Functions}}
\funcdescription{fabs, fabsf\newstd, fabsl\newstd}{absolute value}
\funcdescription{fmod$^{(*)}$\newstd}{remainder of division}
\smallheader{$^{(*)}$\newstd also presented for \texttt{float} and \texttt{long double}, added in C99}
\funcdescription{remainder\newstd $^{(*)}$\newstd}{signed remainder of division}
\funcdescription{remquo\newstd $^{(*)}$\newstd}{signed remainder as well as the three last bits of the division}
\funcdescription{fma\newstd $^{(*)}$\newstd}{fused multiply-add operation $x * y + z$}
\funcdescription{fmax\newstd $^{(*)}$\newstd}{determines larger of two values}
\funcdescription{fmin\newstd $^{(*)}$\newstd}{determines smaller of two values}
\funcdescription{fdim\newstd $^{(*)}$\newstd}{positive difference of two floating-point values $max(0, x - y)$}
\funcdescription{nan\newstd $^{(*)}$\newstd}{returns a NaN (not-a-number)}
\funcdescription{exp$^{(*)}$\newstd}{$e^x$}
\funcdescription{exp2\newstd $^{(*)}$\newstd}{$2^x$}
\funcdescription{expm1\newstd $^{(*)}$\newstd}{$e^x - 1$}
\funcdescription{log$^{(*)}$\newstd}{natural (base-e) logarithm $ln x$}
\funcdescription{log10$^{(*)}$\newstd}{common (base-10) logarithm $log_{10} x$}
\funcdescription{log2\newstd $^{(*)}$\newstd}{base-2 logarithm $log_2 x$}
\funcdescription{log1p\newstd $^{(*)}$\newstd}{$ln (1 + x)$}
\funcdescription{pow$^{(*)}$\newstd}{$x^y$}
\funcdescription{sqrt$^{(*)}$\newstd}{$\sqrt{x}$}
\funcdescription{cbrt\newstd $^{(*)}$\newstd}{$\sqrt[3]{x}$}
\funcdescription{hypot\newstd $^{(*)}$\newstd}{$\sqrt{x^2 + y^2}$}
\funcdescription{sin$^{(*)}$\newstd}{$\sin{x}$}
\funcdescription{cos$^{(*)}$\newstd}{$\cos{x}$}
\funcdescription{tan$^{(*)}$\newstd}{$\tan{x}$}
\funcdescription{asin$^{(*)}$\newstd}{$\arcsin{x}$}
\funcdescription{acos$^{(*)}$\newstd}{$\arccos{x}$}
\funcdescription{atan$^{(*)}$\newstd}{$\arctan{x}$}
\funcdescription{atan2$^{(*)}$\newstd}{$\arctan{x}$ using signs to detect quadrants}
\funcdescription{sinh$^{(*)}$\newstd}{$\sinh x$}
\funcdescription{cosh$^{(*)}$\newstd}{$\cosh x$}
\funcdescription{tanh$^{(*)}$\newstd}{$\tanh x$}
\funcdescription{asinh\newstd $^{(*)}$\newstd}{$\textrm{arcsinh}\ x$}
\funcdescription{acosh\newstd $^{(*)}$\newstd}{$\textrm{arccosh}\ x$}
\funcdescription{atanh\newstd $^{(*)}$\newstd}{$\textrm{arctanh}\ x$}
\funcdescription{erf\newstd $^{(*)}$\newstd}{error function}
\funcdescription{erfc\newstd  $^{(*)}$\newstd}{complementary error function}
\funcdescription{tgamma\newstd $^{(*)}$\newstd}{gamma function}
\funcdescription{lgamma\newstd $^{(*)}$\newstd}{natural logarithm of gamma function}
\funcdescription{ceil$^{(*)}$\newstd}{smallest integer not less than the given value}
\funcdescription{floor$^{(*)}$\newstd}{largest integer not greater than the given value}
\funcdescription{trunc\newstd $^{(*)}$\newstd}{nearest integer not greater in magnitude}
\funcdescription{round\newstd  $^{(*)}$\newstd, lround\newstd $^{(*)}$\newstd, llround\newstd  $^{(*)}$\newstd}{rounds to nearest integer, rounding away from zero in halfway cases }
\funcdescription{nearbyint\newstd $^{(*)}$\newstd}{round to an integer using current rounding mode}
\funcdescription{rint$^{(*)}$\newstd, lrint $^{(*)}$\newstd, llrint $^{(*)}$\newstd}{round to an integer using current rounding mode with
exception if the result differs}
\funcdescription{frexp$^{(*)}$\newstd}{break a number into significand and a power of 2}
\funcdescription{ldexp$^{(*)}$\newstd}{multiply a number by 2 raised to a power}
\funcdescription{modf$^{(*)}$\newstd}{break a number into integer and fractional parts}
\funcdescription{scalbn\newstd $^{(*)}$\newstd, scalbln\newstd $^{(*)}$\newstd}{compute efficiently a number times \texttt{FLT\_RADIX} raised to a power}
\funcdescription{ilogb\newstd $^{(*)}$\newstd}{extract exponent of the given number}
\funcdescription{logb\newstd $^{(*)}$\newstd}{extract exponent of the given number}
\funcdescription{nextafter\newstd $^{(*)}$\newstd, nexttoward\newstd $^{(*)}$\newstd}{next representable floating-point value towards the given value}
\funcdescription{copysign\newstd $^{(*)}$\newstd}{value with the magnitude of a given value and the sign of another given value}
\funcdescription{fpclassify\newstd}{classify the given floating-point value}
\funcdescription{isfinite\newstd}{given number has finite value?}
\funcdescription{isinf\newstd}{number is infinite?}
\funcdescription{isnan\newstd}{number is NaN?}
\funcdescription{isnormal\newstd}{number is normal?}
\funcdescription{signbit\newstd}{number is negative?}
\funcdescription{isgreater\newstd}{first argument is greater than second?}
\funcdescription{isgreaterequal\newstd}{first argument is greater or equal than second?}
\funcdescription{isless\newstd}{first argument is less than second?}
\funcdescription{islessequal\newstd}{first argument is less or equal than second?}
\funcdescription{islessgreater\newstd}{first argument is less or greater than second?}
\funcdescription{isunordered\newstd}{two values are unordered?}
}


\librarysection{Program Support Utilities}{setjmp.h}{0.35}{0.55}{
\smallheader{\underline{Types}}
\funcdescription{jmp\_buf}{execution context type}
\smallheader{\underline{Functions}}
\funcdescription{setjmp}{save context}
\funcdescription{longjmp}{jump to specified location}
}

\librarysection{Program Support}{signal.h}{0.3}{0.6}{
\smallheader{\underline{Types}}
\funcdescription{sig\_atomic\_t}{integer type that can be accessed as an atomic entity from an asynchronous signal handler}
\smallheader{\underline{Macros}}
\funcdescription{SIGABRT, SIGFPE, SIGILL,SIGINT, SIGSEGV, SIGTERM}{signal types}
\funcdescription{SIG\_DFL, SIG\_IGN}{signal handling strategies}
\funcdescription{SIG\_ERR}{error was encountered}
\smallheader{\underline{Functions}}
\funcdescription{signal}{set signal handler for particular signal}
\funcdescription{raise}{run signal handler for particular signal}
}

\librarysection{Variable Argument Lists}{stdarg.h}{0.25}{0.65}{
\smallheader{Function definition: \texttt{type name(t1 arg1, \dots)}}
\smallheader{\underline{Types}}
\funcdescription{va\_list}{information needed by all macros}
\smallheader{\underline{Macros}}
\funcdescription{va\_start}{initialize argument pointer \texttt{ap}, \texttt{lastarg} - last named argument}
\funcdescription{va\_arg}{access next argument}
\funcdescription{va\_copy\newstd}{copy arguments}
\funcdescription{va\_end}{end traversal}
}

\librarysection{Boolean Type}{stdbool.h}{0.45}{0.45}{
\smallheader{\underline{Macros}}
\funcdescription{bool}{boolean type definition}
\funcdescription{true}{integer 1}
\funcdescription{false}{integer 0}
}

\librarysection{Types Support}{stddef.h}{0.2}{0.7}{
\smallheader{\underline{Types}}
\funcdescription{ptrdiff\_t}{signed int, result of two pointers subtraction}
\funcdescription{size\_t}{unsigned int returned by \texttt{sizeof}, \texttt{offsetof}}
\smallheader{\underline{Constants}}
\funcdescription{NULL}{implementation-defined null pointer constant}
\smallheader{\underline{Macros}}
\funcdescription{offsetof}{byte offset from the beginning of a struct type to specified member}
}

\librarysection{Integer Type Support}{stdint.h\newstd}{0.35}{0.55}{
\smallheader{\underline{Types}}
\funcdescription{int8\_t, int16\_t, int32\_t, int64\_t}{signed int with exact width}
\funcdescription{int\_fast8\_t, \dots, int\_fast64\_t}{fastest signed int with width at least 8, 16, \dots}
\funcdescription{int\_least8\_t, \dots, int\_least64\_t}{smallest signed int with width at least 8, 16, \dots}
\funcdescription{intmax\_t}{maximum width integer type}
\funcdescription{intptr\_t}{integer type capable of holding a pointer}
\funcdescription{uint8\_t, \dots, uint64\_t}{unsigned int with exact width}
\funcdescription{uint\_fast8\_t, \dots, uint\_fast64\_t}{fastest unsigned int with width at least 8, 16, \dots}
\funcdescription{uint\_least8\_t, \dots, uint\_least64\_t}{smallest unsigned int with width at least 8, 16, \dots}
\funcdescription{uintmax\_t}{maximum width unsigned integer type}
\funcdescription{uintptr\_t}{unsigned integer type capable of holding a pointer}
\smallheader{\underline{Constants}}
\funcdescription{INT8\_MIN, \dots, INT64\_MIN}{minimum value of object of corresponding type}
\funcdescription{INT\_FAST8\_MIN, \dots, INT\_FAST64\_MIN}{minimum value of object of corresponding type}
\funcdescription{INT\_LEAST8\_MIN, \dots, INT\_LEAST64\_MIN}{minimum value of object of corresponding type}
\funcdescription{INTPTR\_MIN}{minimum value of intptr\_t object}
\funcdescription{INTMAX\_MIN}{minimum value of intmax\_t object}
\funcdescription{INT8\_MAX, \dots, INT64\_MAX}{maximum value of object of corresponding type}
\funcdescription{INT\_FAST8\_MAX, \dots, INT\_FAST64\_MAX}{maximum value of object of corresponding type}
\funcdescription{INT\_LEAST8\_MAX, \dots, INT\_LEAST64\_MAX}{maximum value of object of corresponding type}
\funcdescription{INTPTR\_MAX}{maximum value of intptr\_t object}
\funcdescription{INTMAX\_MAX}{maximum value of intmax\_t object}
\funcdescription{UINT8\_MAX, \dots, UINT64\_MAX}{maximum value of object of corresponding type}
\funcdescription{UINT\_FAST8\_MAX, \dots, UINT\_FAST64\_MAX}{maximum value of object of corresponding type}
\funcdescription{UINT\_LEAST8\_MAX, \dots, UINT\_LEAST64\_MAX}{maximum value of object of corresponding type}
\funcdescription{UINTPTR\_MAX}{maximum value of uintptr\_t object}
\funcdescription{UINTMAX\_MAX}{maximum value of uintmax\_t object}
\smallheader{\underline{Function Macro}}
\funcdescription{INT8\_C, \dots, INT64\_C}{expands to an int const expression with the type int\_least8\_t, \dots}
\funcdescription{INTMAX\_C}{expands to an int const expression with the type intmax\_t}
\funcdescription{UINT8\_C, \dots, UINT64\_C}{expands to an int const expression with the type uint\_least8\_t, \dots}
\funcdescription{UINTMAX\_C}{expands to an int const expression with the type uintmax\_t}
}

\librarysection{Standard input/output}{stdio.h}{0.3}{0.6}{
\smallheader{\underline{Types}}
\funcdescription{FILE}{object type, capable of holding all information needed to control a C I/O stream}
\funcdescription{fpos\_t}{non-array complete object type, capable of uniquely specifying a position and multibyte parser state in a file}
\smallheader{\underline{Predefined standard streams}}
\funcdescription{stdin, stdout, stderr}{expression of type \texttt{FILE *} associated with corresponding stream}
\smallheader{\underline{Functions}}
\funcdescription{fopen}{open file,  returns \texttt{FILE *}}
\funcdescription{freopen}{open an existing stream \texttt{FILE *fp} with a different name, returns \texttt{FILE *}}
\funcdescription{fclose}{close a file}
\funcdescription{fflush}{synchronizes an output stream with the actual file}
\funcdescription{setbuf}{sets the buffer for a file stream}
\funcdescription{setvbuf}{sets the buffer and its size for a file stream}
\funcdescription{fread}{reads from a file \texttt{count} objects of size \texttt{size} to \texttt{buffer}}
\funcdescription{fwrite}{writes to a file \texttt{count} objects of size \texttt{size} from \texttt{buffer}}
\funcdescription{fgetc, getc}{gets a character from a file stream}
\funcdescription{fgets}{gets a character string with length \texttt{count - 1} from a file stream}
\funcdescription{fputc, putc}{writes a character to a file stream}
\funcdescription{fputs}{writes a character string to a file stream}
\funcdescription{getchar}{reads a character from \texttt{stdin}, equivalent to \texttt{getc (stdin)}}
\funcdescription{gets}{reads a character string from \texttt{stdin} until newline or \texttt{EOF}}
\funcdescription{putchar}{writes a character to stdout, equivalent to \texttt{putc (ch, stdout)}}
\funcdescription{puts}{writes a character string + \texttt{\textbackslash{}n} to stdout}
\funcdescription{ungetc}{puts a character back into a file stream}
\funcdescription{scanf, fscanf, sscanf}{reads formatted input from stdin, a file stream or a buffer}
\funcdescription{vscanf\newstd, vfscanf\newstd, vsscanf\newstd}{reads formatted input from stdin, a file stream or a buffer using variable argument list}
\funcdescription{printf, fprintf, sprintf, snprintf\newstd}{prints formatted output to stdout, a file stream or a buffer}
\funcdescription{vprintf, vfprintf, vsprintf, vsnprintf\newstd}{prints formatted output to stdout, a file stream or a buffer using variable argument list}
\funcdescription{ftell}{returns the current file position indicator, useful for \texttt{fseek}}
\funcdescription{fgetpos}{gets the file position indicator, useful for \texttt{fsetpos}}
\funcdescription{fseek}{moves the file position indicator to a specific location in a file, origin values: \texttt{SEEK\_SET}, \texttt{SEEK\_CUR}, \texttt{SEEK\_END}}
\funcdescription{fsetpos}{moves the file position indicator to a specific location in a file}
\funcdescription{rewind}{moves the file position indicator to the beginning in a file}
\funcdescription{clearerr}{clears errors}
\funcdescription{feof}{checks for the end-of-file}
\funcdescription{ferror}{checks for a file error}
\funcdescription{perror}{displays a character string corresponding of the current error to stderr}
\funcdescription{remove}{erases a file}
\funcdescription{rename}{renames a file}
\funcdescription{tmpfile}{returns a pointer to a temporary file}
\funcdescription{tmpnam}{returns a unique filename}
\smallheader{\underline{Macro constants}}
\funcdescription{EOF}{integer constant expression of type int and negative value}
\funcdescription{FOPEN\_MAX}{maximum number of files that can be open simultaneously}
\funcdescription{FILENAME\_MAX}{size needed for an array of char to hold the longest supported file name}
\funcdescription{BUFSIZ}{size of the buffer used by setbuf}
\funcdescription{\_IOFBF, \_IOLBF, \_IONBF}{argument to setvbuf indicating fully buffered, line buffered, unbuffered I/O}
\funcdescription{SEEK\_SET, SEEK\_CUR, SEEK\_END}{argument to fseek indicating seeking from beginning, current position, end of the file}
\funcdescription{TMP\_MAX}{maximum number of unique filenames that can be generated by \texttt{tmpnam}}
\funcdescription{L\_tmpnam}{size needed for an array of char to hold the result of tmpnam}
}

% TODO: r/w modes, printf formats

\librarysection{Standard library}{stdlib.h}{0.25}{0.65}{
\smallheader{\underline{Macros}}
\funcdescription{EXIT\_SUCCESS, EXIT\_FAILURE}{indicates program execution execution status}
\funcdescription{MB\_CUR\_MAX}{maximum number of bytes in a multibyte character, in the current locale}
\funcdescription{RAND\_MAX}{maximum possible value generated by rand()}
\smallheader{\underline{Functions}}
\funcdescription{abort}{causes abnormal program termination (without cleaning up)}
\funcdescription{exit}{causes normal program termination with cleaning up}
\funcdescription{\_Exit\newstd}{causes normal program termination without cleaning up}
\funcdescription{atexit}{registers a function to be called on exit() invocation}
\funcdescription{system}{calls the host environment's command processor }
\funcdescription{getenv}{access to the list of environment variables}
\funcdescription{malloc}{allocates memory}
\funcdescription{calloc}{allocates and zeroes memory}
\funcdescription{realloc}{expands previously allocated memory block}
\funcdescription{free}{deallocates previously allocated memory}
\funcdescription{atof}{converts a byte string to a floating-point value}
\funcdescription{atoi, atol, atoll\newstd}{converts a byte string to an integer value}
\funcdescription{strtol, strtoll\newstd}{converts a byte string to an integer value}
\funcdescription{strtoul, strtoull\newstd}{converts a byte string to an unsigned integer value}
\funcdescription{strtof\newstd, strtod, strtold\newstd}{converts a byte string to a floating-point value }
\funcdescription{mblen}{returns the number of bytes in the next multibyte character}
\funcdescription{mbtowc}{converts the next multibyte character to wide character}
\funcdescription{wctomb}{converts a wide character to its multibyte representation}
\funcdescription{mbstowcs}{converts a narrow multibyte character string to wide string}
\funcdescription{wcstombs}{converts a wide string to narrow multibyte character string}
\funcdescription{rand}{generates a pseudo-random number}
\funcdescription{srand}{seeds pseudo-random number generator}
\funcdescription{qsort}{sorts a range of elements with unspecified type}
\funcdescription{bsearch}{searches an array for an element of unspecified type}
}

\librarysection{NULL-terminated strings}{string.h}{0.2}{0.7}{
\smallheader{\underline{Macros}}
\funcdescription{NULL}{implementation-defined null pointer constant}
\smallheader{\underline{Types}}
\funcdescription{size\_t}{unsigned integer type returned by the sizeof operator}
\smallheader{\underline{Functions}}
\funcdescription{strcpy}{copies one string to another}
\funcdescription{strncpy}{copies a certain amount of characters from one string to another}
\funcdescription{strcat}{concatenates two strings}
\funcdescription{strncat}{concatenates a certain amount of characters of two strings}
\funcdescription{strxfrm}{transform a string so that strcmp would produce the same result as strcoll}
\funcdescription{strlen}{returns the length of a given string}
\funcdescription{strcmp}{compares two strings}
\funcdescription{strncmp}{compares a certain amount of characters of two strings}
\funcdescription{strcoll}{compares two strings in accordance to the current locale}
\funcdescription{strchr}{finds the first occurrence of a character}
\funcdescription{strrchr}{finds the last occurrence of a character}
\funcdescription{strspn}{returns the length of the maximum initial segment that consists of only the characters found in another byte string}
\funcdescription{strcspn}{returns the length of the maximum initial segment that consists of only the characters not found in another byte string}
\funcdescription{strpbrk}{finds the first location of any character in one string, in another string}
\funcdescription{strstr}{finds the first occurrence of a substring of characters}
\funcdescription{strtok}{finds the next token in a byte string}
\funcdescription{memchr}{searches an array for the first occurrence of a character}
\funcdescription{memcmp}{compares two buffers}
\funcdescription{memset}{fills a buffer with a character}
\funcdescription{memcpy}{copies one buffer to another}
\funcdescription{memmove}{moves one buffer to another}
\funcdescription{strerror}{returns a text version of a given error code}
}

\librarysection{Type generic math}{tgmath.h\newstd}{0.45}{0.45}{
\smallheader{Determines, which real or complex function}
\smallheader{to call for the provided arguments}
% FIXME
}

\librarysection{Date and time utilities}{time.h}{0.2}{0.7}{
\smallheader{\underline{Types}}
\funcdescription{tm}{calendar time type}
\funcdescription{time\_t}{calendar time since epoch type}
\funcdescription{clock\_t}{processor time since era type}
\smallheader{\underline{Constants}}
\funcdescription{CLOCKS\_PER\_SEC}{number of processor clock ticks per second}
\smallheader{\underline{Functions}}
\funcdescription{difftime}{computes the difference between times}
\funcdescription{time}{returns the current calendar time of the system as time since epoch}
\funcdescription{clock}{returns raw processor clock time since the program is started}
\funcdescription{asctime}{converts a tm object to a textual representation}
\funcdescription{ctime}{converts a time\_t object to a textual representation}
\funcdescription{strftime}{converts a tm object to custom textual representation}
\funcdescription{gmtime}{converts time since epoch to calendar time expressed as Coordinated Universal Time (UTC)}
\funcdescription{localtime}{converts time since epoch to calendar time expressed as local time}
\funcdescription{mktime}{converts calendar time to time since epoch}
}

\librarysection{Wide character support}{wchar.h\newstd}{0.2}{0.7}{
\smallheader{\underline{Types}}
\funcdescription{mbstate\_t}{conversion state information necessary to iterate multibyte character strings}
\funcdescription{wchar\_t}{integer type that can hold any valid wide character}
\funcdescription{wint\_t}{integer type that can hold any valid wide character and at least one more value}
\smallheader{\underline{Macros}}
\funcdescription{WEOF}{a non-character value of type \texttt{wint\_t} used to indicate errors}
\funcdescription{WCHAR\_MIN}{the smallest valid value of \texttt{wchar\_t}}
\funcdescription{WCHAR\_MAX}{the largest valid value of \texttt{wchar\_t}}
\smallheader{\underline{Functions}}
\funcdescription{fwide}{switches a file stream between wide character I/O and narrow character I/O}
\funcdescription{fgetwc, getwc}{gets a wide character from a file stream}
\funcdescription{fgetws}{gets a wide string of length \texttt{count - 1} from a file stream}
\funcdescription{fputwc, putwc}{writes a wide character to a file stream}
\funcdescription{fputws}{writes a wide string to a file stream}
\funcdescription{getwchar}{reads a wide character from stdin}
\funcdescription{putwchar}{writes a wide character to stdout}
\funcdescription{ungetwc}{puts a wide character back into a file stream}
\funcdescription{wscanf, fwscanf, swscanf}{reads formatted wide character input from stdin, a file stream or a buffer}
\funcdescription{vwscanf, vfwscanf, vswscanf}{reads formatted wide character input from stdin, a file stream or a buffer using variable argument list}
\funcdescription{wprintf, fwprintf, swprintf}{prints formatted wide character output to stdout, a file stream or a buffer}
\funcdescription{vwprintf, vfwprintf, vswprintf}{prints formatted wide character output to stdout, a file stream or a buffer using variable argument list}
\funcdescription{mbsinit}{checks if the mbstate\_t object represents initial shift state}
\funcdescription{btowc}{widens a single-byte narrow character to wide character, if possible}
\funcdescription{wctob}{narrows a wide character to a single-byte narrow character, if possible}
\funcdescription{mbrlen}{returns the number of bytes in the next multibyte character, given state}
\funcdescription{mbrtowc}{converts the next multibyte character to wide character, given state}
\funcdescription{wcrtomb}{converts a wide character to its multibyte representation, given state}
\funcdescription{mbsrtowcs}{converts a narrow multibyte character string to wide string, given state}
\funcdescription{wcsrtombs}{converts a wide string to narrow multibyte character string, given state}
\funcdescription{wcstol, wcstoll}{converts a wide string to an integer value}
\funcdescription{wcstoul, wcstoull}{converts a wide string to an unsigned integer value}
\funcdescription{wcstof, wcstod, wcstold}{converts a wide string to a floating-point value}
\funcdescription{wcscpy}{copies one wide string to another}
\funcdescription{wcsncpy}{copies a certain amount of wide characters from one string to another}
\funcdescription{wcscat}{appends a copy of one wide string to another}
\funcdescription{wcsncat}{appends a certain amount of wide characters from one wide string to another}
\funcdescription{wcsxfrm}{transform a wide string so that \texttt{wcscmp} would produce the same result as \texttt{wcscoll}}
\funcdescription{wcslen}{returns the length of a wide string}
\funcdescription{wcscmp}{compares two wide strings}
\funcdescription{wcsncmp}{compares a certain amount of characters from two wide strings}
\funcdescription{wcscoll}{compares two wide strings in accordance to the current locale}
\funcdescription{wcschr}{finds the first occurrence of a wide character in a wide string}
\funcdescription{wcsrchr}{finds the last occurrence of a wide character in a wide string}
\funcdescription{wcsspn}{returns the length of the maximum initial segment that consists of only the wide characters found in another wide string}
\funcdescription{wcscspn}{returns the length of the maximum initial segment that consists of only the wide chars not found in another wide string}
\funcdescription{wcspbrk}{finds the first location of any wide character in one wide string, in another wide string}
\funcdescription{wcsstr}{finds the first occurrence of a wide string within another wide string}
\funcdescription{wcstok}{finds the next token in a wide string}
\funcdescription{wmemcpy}{copies a certain amount of wide characters between two non-overlapping arrays}
\funcdescription{wmemmove}{copies a certain amount of wide characters between two, possibly overlapping, arrays}
\funcdescription{wmemcmp}{compares a certain amount of wide characters from two arrays}
\funcdescription{wmemchr}{finds the first occurrence of a wide character in a wide character array}
\funcdescription{wmemset}{copies the given wide character to every position in a wide character array}
}

\librarysection{Wide character support}{wctype.h\newstd}{0.2}{0.7}{
\smallheader{\underline{Types}}
\funcdescription{wint\_t}{integer type that can hold any valid wide character and at least one more value}
\funcdescription{wctrans\_t}{scalar type that holds locale-specific character mapping}
\funcdescription{wctype\_t}{scalar type that holds locale-specific character classification}
\smallheader{\underline{Macros}}
\funcdescription{WEOF}{a non-character value of type \texttt{wint\_t} used to indicate errors}
\smallheader{\underline{Functions}}
\funcdescription{iswalnum}{checks if a wide character is alphanumeric}
\funcdescription{iswalpha}{checks if a wide character is alphabetic}
\funcdescription{iswlower}{checks if a wide character is an lowercase character}
\funcdescription{iswupper}{checks if a wide character is an uppercase character}
\funcdescription{iswdigit}{checks if a wide character is a digit}
\funcdescription{iswxdigit}{checks if a wide character is a hexadecimal character}
\funcdescription{iswcntrl}{checks if a wide character is a control character}
\funcdescription{iswgraph}{checks if a wide character is a graphical character}
\funcdescription{iswspace}{checks if a wide character is a space character}
\funcdescription{iswblank}{checks if a wide character is a blank character}
\funcdescription{iswprint}{checks if a wide character is a printing character}
\funcdescription{iswpunct}{checks if a wide character is a punctuation character}
\funcdescription{iswctype}{classifies a wide character according to the specified texttt{LC\_CTYPE} category}
\funcdescription{wctype}{looks up a character classification category in the current C locale}
\funcdescription{towlower}{converts a wide character to lowercase}
\funcdescription{towupper}{converts a wide character to uppercase}
\funcdescription{towctrans}{performs character mapping according to the specified \texttt{LC\_CTYPE} mapping category}
\funcdescription{wctrans}{looks up a character mapping category in the current C locale}
}

\end{multicols*}
\end{document}
